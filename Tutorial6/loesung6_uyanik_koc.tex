\documentclass{pi2}

\makeatletter
\def\input@path{{src/de/uni_bremen/pi2/}{test/de/uni_bremen/pi2}}
\makeatother

\begin{document}

\maketitle{6}{Tobias Diehl}{Altug Uyanik,Öykü Koç}
\section*{Aufgabe 1 Spannende Vernetzung}

% TODO: \usepackage{graphicx} required
\begin{figure}[h]
	\centering
	\includegraphics[width=1\linewidth]{"Screenshot_1.png"}
	\caption{}
	\label{fig:Screenschot_1}
\end{figure}
\section*{Aufgabe 2 Sommer, Sonne, Routenplaner}
\section*{Aufgabe 2.1 Karte aufbauen}

Wir haben die mitgelieferte Map-Klasse so erweitert, dass ihr Konstruktor die Dateien nodes.txt und edges.txt einliest und daraus einen Graphen erstellt. Dazu haben wir auch die mitgelieferten Klassen Node und Edge verwendet. Die Kanten in der Datei edges.txt sind ungerichtet, also haben wir sie in beide Richtungen in unseren Graphen eingetragen. Zum Zeichnen der Karte haben wir die Methode draw verwendet. Nach dem Start des RoutePlanner-Programms wird die Karte nun im Fenster angezeigt.


\lstinputlisting[firstnumber=18,firstline=18,lastline=95]{Map.java}

\section*{Aufgabe 2.2 Positionen wählen}
Es soll eine Methode \texttt{getClosest()} implementiert werden, die den nächstgelegenen Knoten zu einer gegebenen Position zurückgibt. Die Methode erhält als Parameter die x- und y-Koordinaten der Position. Zunächst wird ein Hilfsknoten namens \texttt{position} mit den übergebenen Koordinaten erstellt. Zudem werden Hilfsvariablen \texttt{minDistance} und \texttt{closestNode} initialisiert. Anschließend wird über die Liste der Knoten (\texttt{Nodes}) iteriert. In jeder Iteration wird die Distanz von dem aktuellen Knoten zu der übergebenen Position berechnet. Der Knoten mit der geringsten Distanz wird in der Variable \texttt{closestNode} gespeichert. Am Ende wird dieser Knoten zurückgegeben.
\lstinputlisting[firstnumber=97,firstline=97,lastline=119]{Map.java}

\section*{Aufgabe 2.3 Routenplanung}

Implementiert die shortestPath-Methode der RoutePlanner-Klasse als Suche nach dem kürzesten Weg nach Dijkstra. Zeichnet die gesuchten Kanten, zum Beispiel in blau. Zeichnet den kürzesten Weg in einer anderen Farbe, z. B. rot. 

\lstinputlisting[firstnumber=235,firstline=235,lastline=249]{RoutePlanner.java}

Nun beginnt eine while-Schleife, die solange läuft, bis der Zielknoten ein Teil der ArrayList chosen ist, was bedeutet, dass der schnellste Weg gefunden wurde.

\lstinputlisting[firstnumber=251,firstline=251,lastline=273]{RoutePlanner.java}
Abschließend geht man rekursiv den kürzesten Weg zurück und zeichnet diesen rot in die Karte ein, solange der Vorgänger auf dem Weg vom Startknoten zu diesem Knoten nicht der Endknoten selbst ist.
\lstinputlisting[firstnumber=275,firstline=275,lastline=282]{RoutePlanner.java}
\end{document}
