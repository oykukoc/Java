\documentclass{pi1}
\usepackage{graphicx}
\usepackage[T1]{fontenc}
\begin{document}

% \maketitle{Übungsblatt}{Tutor:in}{Bearbeiter:in}
\maketitle{12}{Pascal Himmelmann}{Öykü Koç}


\textbf{Das Ziel}

Diese Aufgabe soll den Beginn der Multiplayer-Version unseres Spiels markieren.\\
\\Die Bewegung einer Figur in einer Instanz des Spiels wird auf eine Figur in einer zweiten Instanz übertragen, die über das Netzwerk verbunden ist. Alle Aktionen des steuernden Charakters werden auf den gesteuerten Charakter übertragen, so dass sich beide Charaktere auf die gleiche Weise verhalten. \\
\\Die Netzwerkverbindung wird zwischen einer Instanz der Klassen RemotePlayer und ControlledPlayer hergestellt. Das Spiel wird gestartet, indem eine Kopie des Projekts und zwei verschiedene BlueJ-Instanzen von sich selbst ausgeführt werden. Beim Ausführen der ersten Instanz wird der IP-Adressparameter auf null oder \texttt{"}\texttt{"} gesetz. Beim Ausführen der zweiten Instanz wird der IP-Adressparameter auf "127.0.0.1" \texttt{} oder "localhost"\texttt{} gesetzt. Die Portnummer muss in beiden Beispielen identisch sein. Der Wert 5000 kann als Portnummer verwendet werden.

\section{Fernsteuernd}

Die RemotePlayer-Klasse ist von der Player-Klasse abgeleitet.


\lstinputlisting[firstnumber=12,firstline=12,lastline=12]{RemotePlayer.java}

Zusätzlich zu den Parametern der Grundklasse erhält der Konstruktor dieser Klasse die IP-Adresse und die Portnummer des Computers des Zeichners als Parameter
\lstinputlisting[firstnumber=26,firstline=26,lastline=26]{RemotePlayer.java}

Das Kommunikationsnetzwerk wird über den im Konstruktor erstellten Socket vorbereitet. Wenn die ControlledPlayer-Instanz für die Netzwerkumgebung geöffnet ist, werden IP-Adresse und Portnummer veröffentlicht, und die RemotePlayer-Verbindung zur entsprechenden Instanz kann hergestellt werden. Bei Ausnahmen, die während der Verbindungsphase auftreten, wird eine Fehlermeldung zurückgegeben und der Player wird ausgeblendet.

\lstinputlisting[firstnumber=26,firstline=26,lastline=38]{RemotePlayer.java}

Diese Klasse überschreibt die act()-Methode. 
Die Richtungs- und Bewegungsaktionen des Spielers werden über die Tastatur ausgeführt. Die Richtungsinformation des Spielers wird dann über den Socket an die ControlledPlayer-Instanz weitergegeben. Die Richtungsinformation besteht aus den Werten 0, 1, 2 und 3. Die Methode flush() wird aufgerufen, um die Richtungsdaten sofort in den Puffer zu übertragen. Bei Ausnahmen, die während der Übertragungsphase auftreten, wird eine Fehlermeldung zurückgegeben und der Spieler ausgeblendet.


\lstinputlisting[firstnumber=45,firstline=45,lastline=57]{RemotePlayer.java}

Diese Klasse überschreibt die Methode setVisible(final boolean visible). 
Wenn der Spieler unsichtbar ist, wird das Spiel beendet, indem der Socket geschlossen wird, falls er offen ist.

\lstinputlisting[firstnumber=65,firstline=65,lastline=79]{RemotePlayer.java}


\section{Ferngesteuert}

Eine ControlledPlayer-Klasse ist von der Player-Klasse abgeleitet

\lstinputlisting[firstnumber=12,firstline=12,lastline=12]{ControlledPlayer.java}

Der Konstruktor dieser Klasse erhält als Parameter zusätzlich zu den Parametern der Basisklasse die Portnummer des Ports, auf dem die Netzwerkverbindung genutzt werden soll.

\lstinputlisting[firstnumber=31,firstline=31,lastline=31]{ControlledPlayer.java}

Im Konstruktor werden der Server und das Kommunikationsnetzwerk durch ServerSocket und socket vorbereitet. Die ControlledPlayer-Instanz öffnet die Netzwerkumgebung über die als Parameter angegebene Portnummer und nimmt die Verbindung von der RemotePlayer-Instanz an. Sie gibt eine Fehlermeldung für die Ausnahme zurück, die während der Verbindungsphase auftritt, und blendet das Zeichen aus.\\

\lstinputlisting[firstnumber=31,firstline=31,lastline=43]{ControlledPlayer.java}

Diese Klasse überschreibt die act()-Methode. 
Die Richtungsinformationen des Spielers, die mit der RemotePlayer-Instanz über das Netzwerk gesendet werden, werden über den Socket gelesen und dem Charakter in der ControlledPlayer-Instanz zugewiesen. Entsprechend der vom Spieler empfangenen Richtungsinformation erhält die Figur die Richtung und Bewegung. Für Ausnahmen, die auftreten, wenn die Verbindung beendet wird oder die Richtungsinformationen nicht gelesen werden können, werden Fehlermeldungen zurückgegeben, und der Charakter wird ausgeblendet. Wenn die Verbindung beendet wird, wird der Wert als (-1) gelesen.

\lstinputlisting[firstnumber=53,firstline=53,lastline=74]{ControlledPlayer.java}

Diese Klasse überschreibt die Methode setVisible(final boolean visible). 
Wenn die Figur unsichtbar ist, wird das Spiel beendet, indem der Socket geschlossen wird, falls er offen ist.

\lstinputlisting[firstnumber=82,firstline=82,lastline=96]{ControlledPlayer.java}

\section{Spielend}

Zusätzlich zum Basisparameter nimmt der Konstruktor der Klasse Level die Parameter IP-Adresse und Portnummer des Computers des kontrollierten Charakters entgegen.

\lstinputlisting[firstnumber=57,firstline=57,lastline=57]{Level.java}

Sie erstellt eine Spielerinstanz entsprechend den IP-Adress- und Portnummern-Daten aus der Hauptmethode. Auf diese Weise hat das Spiel eine parametrische und flexible Struktur.

Wenn die IP-Adresse vom Typ String, die als Parameter der Konstruktormethode übergeben wird, null oder leer ist;
Eine Spielfigur des kontrollierten Typs wird von der Klasse ControlledPlayer erstellt.

Wenn die IP-Adresse vom Typ String, die als Parameter der Konstruktormethode übergeben wird, voll ist;
Eine Spielfigur vom Typ Controller wird von der Klasse RemotePlayer erstellt

\lstinputlisting[firstnumber=85,firstline=85,lastline=88]{Level.java}

Die Parameter IP-Adresse und Portnummer werden der Hauptmethode der Klasse PI1Game hinzugefügt, um sie an die Klasse Level zu senden

\lstinputlisting[firstnumber=21,firstline=21,lastline=21]{PI1Game.java}

Die entsprechenden Parameter werden als Parameter an die Klasse Level gesendet, wenn eine Instanz der Klasse Level erzeugt wird.

\lstinputlisting[firstnumber=21,firstline=21,lastline=38]{PI1Game.java}










\end{document}

